\documentclass[landscape]{sciposter}

% \title{\fontfamily{librebaskervillefamily}\selectfont Fighting Documentation Drift}
\title{\fontfamily{LinuxLibertineT-OsF}\selectfont Fighting Documentation Drift}
\author{Terrence Reilly}
\email{terrencepreilly@gmail.com}

\usepackage{multicol}
% \usepackage{sectionbox}
\usepackage{minted}
\usepackage{parskip}

% \usepackage{ebgaramond}

% \usepackage{librebaskerville}
\usepackage{libertine}
\usepackage[defaultmono]{droidmono}
% \renewcommand{\familydefault}{\rmdefault}

\setminted{fontsize=\small} % ,baselinestretch=1}
% \setmargins[3cm]

\definecolor{BoxCol}{rgb}{0.82,0.77,0.91}
\linespread{1.15}

\begin{document}
\conference{{\bf Pycon 2018}, poster Session, 13 May 2018, Cleveland, Ohio, USA}
\maketitle
\begin{multicols}{4}

    \section{The Problem}
        The tendency for documentation to become out-of-date with
        the code it describes is known as \textit{documentation drift}.
        Documentation drift can result in lost time, incorrect use of APIs,
        and frustration.

        The following example of documentation drift is taken from
        a convenience method in the project \textit{darglint2}.  The parameter
        \textit{filename} was added to the function, but the docstring
        was not updated.  In this particular example, it is not entirely
        obvious how \textit{filename} is supposed to be used simply from
        reading the function.

\begin{minted}{python}
  def get_error_report(self, verbosity, filename):
      """Return a string representation of the errors.

      Args:
          verbosity:
              The level of verbosity.  Should be an
              integer in the range [1,3].

      Returns:
          A string representation of the errors.

      """
      return str(ErrorReport(
          errors=self.errors,
          filename=filename,
          verbosity=verbosity,
      ))
\end{minted}

    \section{Typical Solutions to Documentation Drift}
        As with all risks, the dangers of documentation drift can be
        mitigated using three general strategies: assessment, avoidance,
        and control.  The most common strategy used by developers is
        avoidance.

        On the extreme end, developers can avoid out-of-sync documentation
        by not writing documentation to begin with.  Proponents of avoidance
        argue that documentation is stale the moment it is written, and
        that code is the only source of truth.  Advocates frequently argue
        that code should be self-documenting, and that the need for
        documentation is a code smell.

        While clarity is a good quality for software to have, relying solely
        on source code as documentation leaves out the benefits of
        documentation (for example, demonstrating intent or describing data
        restrictions/format.)

        Research has show that even out-of-sync documentation can provide
        benefit.  In particular, high-level descriptions of architecture
        and intent age far better than implementation specifics.  Therefore,
        we can effectively avoid most of the threat of documentation drift
        by maintaining high-level documents, and by documenting primarily
        intent.

    \section{Where Avoidance Fails}
         If we would like to document an external API, avoidance does not
         provide a good strategy.  For example, if we avoid documenting
         parameters in a publicly-exposed function, the user will have to
         read the source to understand what values to pass to a given function.
         This can be non-trivial if the parameters are tramp data (that is,
         if they are not used in the function being called, but are passed
         to some other function.)

    \section{Darglint2}

    \section{General Purpose}
        \textit{Darglint2} was written to provide an alternative to the
        avoidance strategy.  Darglint2 helps prevent documentation drift
        by warning developers when docstrings do not match the actual
        function/method body.

        \subsection{Example}
            Let's say we have a base error class which defines a method,
            `message`.  A programmer recently decided that the class should
            not raise exceptions when run in production.  To this end,
            they added a parameter, but forgot to document it:

\begin{minted}{python}
  def message(self, verbosity=1, raises=True):
      """Get the message for this error.

      Args:
          verbosity: An integer in the set {1,2},
              where 1 is a more terse message, and
              2 includes a general description.

      Raises:
          Exception: If the verbosity level is not
              recognized.

      Returns:
          An error message.

      """
      if verbosity == 1:
          return '{}'.format(self.terse_message)
      elif verbosity == 2:
          return '{}: {}'.format(
              self.general_message,
              self.terse_message,
          )
      else:
          if raises:
              raise Exception(
                  'Unrecognized verbosity setting '
                  + str(verbosity)
              )
\end{minted}

            Running \textit{darglint2}, below gives us a warning about the missing
            parameter:

\begin{minted}{bash}
  > darglint2 -v 2 base_error.py
  base_error.py:message:4: I101: Missing parameter(s)
      in Docstring: - raises
\end{minted}

            We now know that a parameter description was not updated with
            the method definition. \\

    \section{Features}

        \textit{Darglint2} includes many features which one would normally expect
        of a linter:

        \subsection{Error silencing}
            Errors can be silenced in two ways: using a configuration file,
            or by including a "noqa" statement in the docstring.  For example,
            the below function would normally raise an exception because
            there is a missing "Returns" section.  But we have prevented the
            error from occurring by including a "noqa".  A bare "noqa"
            statement causes all errors to be ignored for the docstring.

\begin{minted}{python}
  def get_user_config():
      """Returns the configuration.

      # noqa: I201

      """
      from django.conf import settings
      return settings.USER_SETTINGS
\end{minted}

        \subsection{Message templating}
            A template can be passed to \textit{darglint2} for formatting error
            messages.  This can be useful for automatic processing.  For
            example, the below call only prints the error number:

\begin{minted}{bash}
  darglint2 --message-template "{msg_id}" *.py
\end{minted}

        \subsection{Syntax errors}
            Errors will also be raised for problems with syntax.  This
            makes it easier to ensure a consistent format and to make
            docstrings parsable for \textit{darglint2}.

        \subsection{Verbosity settings}
            There are two default verbosity settings more or less in-depth
            error messages.  For example, the function
            \begin{minted}{python}
  def insert(self, item):
      """Insert the given item into this binary tree."""
      ...
            \end{minted}

            Will produce the following output from darglint2:

            \begin{minted}{bash}
  > darglint2 -v 1 binary_tree.py
  binary_tree.py:insert:11: I101: - item

  > darglint2 -v 2 binary_tree.py
  binary_tree.py:insert:11: I101: Missing parameter(s) in
      Docstring: - item

            \end{minted}


        \subsection{Type comparisons}
            When converting a project over to use type hints, it is easy
            for documentation to fall behind.  If the docstring uses Google-doc
            style types, then it will compare them against the type signature
            and raise an error if they do not match. For example, the below
            function will raise the error ``add\_two.py:add\_two:1: I103: ~y:
            expected int but was float''.

\begin{minted}{python}
  def add_two(x: int, y: int) -> int:
      """Add two numbers.

      Args:
          x (int): The first number.
          y (float): The second number.

      Returns:
          int: The sum of the first and second number.

      """
      return x + y
\end{minted}

    \section{Disadvantages}
        The primary disadvantage of \textit{darglint2} is that it only warns
        missing or extra parameters, return statments, etc.  If you change how
        a parameter is used, or its order, then \textit{darglint2} is of no help.

    \section{Alternatives}
        There are many alternative strategies for mitigating documentation
        drift.  One popular strategy for avoiding documentation drift is
        to extract documentation directly from the code.  This is done,
        for example, by Swagger to generate descriptions of APIs using
        the OpenAPI specification.  This, of course, is limited as it does
        not describe intent; it is a more convenient means of viewing the
        function definitions.

        While avoidance is common, and \textit{darglint2} attempts to use control
        to prevent documentation drift, there are relatively few attempts
        at using assessment as a preventative/remedial measure.  One possibility
        entertained by a commenter on Reddit (/u/robertmeta) is to use
        version control to see whether comments are likely to be out-of-date:

        \begin{quote}
            It got so bad on a codebase I worked on a young dev who joined
            the team wrote a little git utility to check the function comment
            to function body changes and mark them all with date diffs and
            "absolute lie" "probably a lie" "maybe true" "probably true".
        \end{quote}

        A more sophisticated utility could identify code which has been
        altered more recently than the documentation which describes it.
        You could then order all documentation in a code base by how out-of-sync
        it is likely to be.  Such a Pareto chart would make a good starting
        point for documentation review.

        A final idea is to tie documentation to some specific set of tests.
        This is the idea behind \textit{literate\_integration}, a small framework
        for writing integration tests designed to be run and generated
        into documentation. (The idea is based on the concept of literate
        programming from Donald Knuth.)  In \textit{literate\_integration}, you
        write from which you can generate documentation in Markdown.  If
        the test fails, then the documentation is out of date and must
        be updated.  Because the documentation is strongly coupled with
        the test, updating the test makes it more likely for the documentation
        to be up to date.

\end{multicols}

\begin{thebibliography}{9}
% I've included some relevant facts from the papers for this
% poster.  Not all of the facts were used; mostly those which
% were repeated across multiple papers.

  \bibitem{satish16}
    C. J. Satish and M. Anand
    \textit{Software Documentation Management Issues and Practices: a Survey}
    Indian Journal of Science and Technology
    Tamilnadu, India
    May 2016
    % It has been found that even documents that are
    % not up to date may still be useful. The value of documenta-
    % tion is judged based on its capability to impart knowledge
    % to its users even when it’s not up to date

  \bibitem{lethbridge03}
    Timothy C. Lethbridge and Janice Singer and Andrew Forward
    \textit{How Software Engineers Use Documentation: The State of the Practice}
    IEEE Software
    November/December 2003
    % Architecture and other abstract documentation information is
    % often valid or at least provides historical guidance that can
    % be useful for maintainers.

    % Systems often have too much documentation.

    % The correlation between a document type’s perceived accuracy and
    % its consultation frequency
    % Document type             Correlation
    % Testing or quality        0.67 (p < .005)
    % Low-level design          0.58 (p < .005)
    % Requirements              0.43 (p < .05)
    % Architectural             0.41 (p < .05)
    % Detailed design           0.39 (p < .05)
    % Specifications            0.03

\end{thebibliography}
\end{document}

